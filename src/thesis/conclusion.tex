\chapter{Conclusion and Future Scope}
\label{conclusion}

\section{Future Research Scope}
\label{conclusion:future-scope}
\subsection{Grannular Table Type Classification}
\label{conclusion:future-scope:type-class}

The table types defined by \cite{kim2012scientific} are general purpose descriptions which can apply to various domains. Although they are quite useful, tables specifically in recent CS literature can have more grannular sub-classifications. For example a Statistics Table can be describing a Dataset for Machine Learning experiments. The tables which may seem as Experiments Results can also be describing an Ablation Study conducted on a model. Due to a very nuanced distinction between a lot of these tables, The task of machine learning to classify table types needs more grannular sub-types within the classes defined by \cite{kim2012scientific}. 

Multiple grannular table types such as Ablation Study, Dataset Descriptions, can be further described and datasets