\chapter{Proposed Solutions}
\label{method}

\section{Sci-Genie : Search Engine Harnessing Structure Of Scientific Research}
\begin{figure}[h]
    \centering
    \includegraphics[width=\maxwidth{\textwidth}]{src/images/sci-genie-context-exp.png}
    \caption{Search Results With Context For Query \textit{pytorch, transformer, adversarial examples} Using Sci-Genie}
    \label{figure\arabic{figurecounter}}
\end{figure}
\refstepcounter{figurecounter}

Sci-Genie is a prototype search engine over CS ArXiv. 
Sci-Genie indexes the hierarchical structure of Computer Science research papers 
to provide more context to search results fragments. 
Sci-Genie's content mining engine also classifies a paper's ontology using a SOTA ontology classification model
developed by \cite{salatino2020ontology}. The mined ontology helps create a "Controled Vocabulary" for filtering 
search results. 

\section{Sci-Genie-Extension: In Browser Extension To Augment Information About Research }

\begin{figure}[h]
    \centering
    \includegraphics[width=\maxwidth{\textwidth}]{src/images/sci-genie-ext-exp.png}
    \caption{ Sci-Genie Browser Plugin}
    \label{figure\arabic{figurecounter}}
\end{figure}
\refstepcounter{figurecounter}

Sci-Genie at its core holds research from CS ArXiv, Tables from the paper mined from CS ArXiv and 
6.5M papers filtered from the Semantic Scholar Open Research Corpus\parencite{ammar-etal-2018-construction}.
The research from the Semantic Scholar Open Research Corpus helps create a citation graph which supports 
information augmentation for the browser plugin. 

\begin{figure}[h]
    \centering
    \includegraphics[width=\maxwidth{\textwidth}]{src/images/sci-genie-ext-table-exp.png}
    \caption{Sci-Genie Browser Plugin surfacing context information about tables from cited papers }
    \label{figure\arabic{figurecounter}}
\end{figure}
\refstepcounter{figurecounter}


\begin{figure}[h]
    \includegraphics[width=0.475\textwidth]{src/images/sci-genie-ext-cite-out-exp.png}
    \hfill
    \includegraphics[width=0.475\textwidth]{src/images/sci-genie-ext-cite-exp.png}
    \caption{ Sci-Genie Browser Plugin surfacing context information about papers citing/cited-by the current paper }
    \label{figure\arabic{figurecounter}}
\end{figure}
\refstepcounter{figurecounter}

The citation graph helps create surface context relevant tables from papers cited by the paper the user is reading. 
Figure \ref{figure9} shows and example of Sci-Genie surfacing tables from the neighborhood of papers. The plugin also 
supports filtering tables which may show comparison of entities using a machine learning model. The citation graph helps
extract papers citing the current paper and papers cited by the current paper as seen in Figure \ref{figure10}; The search 
engine helps surface other papers written by the authors of the paper the user is reading. 

All information provided by the browser-extension, helps enhance the understanding of the researcher without the researcher 
having to dig up the information on their own. 

\begin{figure}[h]
    \centering
    \includegraphics[width=\maxwidth{\textwidth}]{src/images/sci-genie-ext-authors-exp.png}
    \caption{ Sci-Genie Browser Plugin surfacing context information about other papers from the authors of the paper the user is reading}
    \label{figure\arabic{figurecounter}}
\end{figure}
\refstepcounter{figurecounter}

\subsection{Design Choices}
TODO : Why Do we choose ArXiv
TODO : Why do we use LateX 
TODO : We explicitly consider PDF to be out of scope.  